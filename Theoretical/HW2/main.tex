\documentclass[a4paper]{article}
\usepackage{ctex}
\usepackage[affil-it]{authblk}
\usepackage[backend=bibtex,style=numeric]{biblatex}
\usepackage{amsmath,amsthm,amssymb,amsfonts}
\usepackage{bm}
\usepackage{pifont}

\usepackage{geometry}
\geometry{margin=1.5cm, vmargin={0pt,1cm}}
\setlength{\topmargin}{-1cm}
\setlength{\paperheight}{29.7cm}
\setlength{\textheight}{25.3cm}

\addbibresource{citation.bib}

\begin{document}
% =================================================
\title{有限元方法 2025 秋冬作业二}

\author{曾申昊 3220100701
  \thanks{邮箱: \texttt{3097714673@qq.com}
                            \texttt{或} \texttt{3220100701@zju.edu.cn}}}
\affil{电子科学与技术 2202,浙江大学 }


\date{更新时间: \today}

\maketitle


% ============================================
\section*{1. 手算有限元刚度矩阵和局部刚度矩阵}

先计算每个单元的局部刚度矩阵 $A_k$,其定义为

\begin{equation}
    A_k =
    \begin{pmatrix}
        \int_{T_k}\nabla \phi_{k_1} \cdot \nabla \phi_{k_1} \text{d}x &
        \int_{T_k}\nabla \phi_{k_1} \cdot \nabla \phi_{k_2} \text{d}x &
        \int_{T_k}\nabla \phi_{k_1} \cdot \nabla \phi_{k_3} \text{d}x\\
        \int_{T_k}\nabla \phi_{k_2} \cdot \nabla \phi_{k_1} \text{d}x &
        \int_{T_k}\nabla \phi_{k_2} \cdot \nabla \phi_{k_2} \text{d}x &
        \int_{T_k}\nabla \phi_{k_2} \cdot \nabla \phi_{k_3} \text{d}x\\
        \int_{T_k}\nabla \phi_{k_3} \cdot \nabla \phi_{k_1} \text{d}x &
        \int_{T_k}\nabla \phi_{k_3} \cdot \nabla \phi_{k_2} \text{d}x &
        \int_{T_k}\nabla \phi_{k_3} \cdot \nabla \phi_{k_3} \text{d}x
    \end{pmatrix}
    \label{局部刚度矩阵}
\end{equation}

根据对称性,我们只需计算三角形 \ding{172} 和 \ding{173} 的局部刚度矩阵。

\begin{equation}
    A_1 =
    \begin{pmatrix}
        \frac{1}{2} & 0 & -\frac{1}{2}\\
        0 & \frac{1}{2} & -\frac{1}{2}\\
        -\frac{1}{2} & -\frac{1}{2} & 1
    \end{pmatrix}, \quad
    A_2 =
    \begin{pmatrix}
        \frac{1}{2} & -\frac{1}{2} & 0\\
        -\frac{1}{2} & 1 & -\frac{1}{2}\\
        0 & -\frac{1}{2} & \frac{1}{2}
    \end{pmatrix}
\end{equation}

将 8 个单元的局部刚度矩阵组装成全局刚度矩阵 $A$

\begin{equation}
    A =
    \begin{pmatrix}
        1 & -\frac{1}{2} & 0 & -\frac{1}{2} & 0 & 0 & 0 & 0 & 0\\
        -\frac{1}{2} & 1 & -\frac{1}{2} & 0 & -\frac{1}{2} & 0 & 0 & 0 & 0\\
        0 & -\frac{1}{2} & 1 & 0 & 0 & -\frac{1}{2} & 0 & 0 & 0\\
        -\frac{1}{2} & 0 & 0 & 1 & -\frac{1}{2} & 0 & -\frac{1}{2} & 0 & 0\\
        0 & -\frac{1}{2} & 0 & -\frac{1}{2} & 2 & -\frac{1}{2} & 0 & -\frac{1}{2} & 0\\
        0 & 0 & -\frac{1}{2} & 0 & -\frac{1}{2} & 1 & 0 & 0 & -\frac{1}{2}\\
        0 & 0 & 0 & -\frac{1}{2} & 0 & 0 & 1 & -\frac{1}{2} & 0\\
        0 & 0 & 0 & 0 & -\frac{1}{2} & 0 & -\frac{1}{2} & 1& -\frac{1}{2}\\
        0& 0& 0& 0& 0& -\frac{1}{2}& 0& -\frac{1}{2}& 1
    \end{pmatrix}
\end{equation}

\section*{2. 二维区域中三角形网格$\mathcal{T}_h$及分片一次有限元空间$V_h$}

\begin{enumerate}
    \item[(a)] 我们知道有限元解 $u_h = \sum_{i=1}^{NV}u_i\phi_i$,
        其中 $\phi_i\in V_h$ 是与节点 $i$ 相关的基函数,
        $u_i$ 是通过解方程$A\vec{U}=\vec{F}$得到的有限元解在节点 $i$ 处的值。\\
        由题可知,三角形 $T$ 三个顶点处的取值均为 $0$。
        由于基函数的局部性以及线性插值的性质可知,
        \begin{equation}
            u_h \big|_T = \sum_{i=1}^{NV}u_i\phi_i \big|_T 
                   = \sum_{i=1}^{3}u_{a_i}\phi_{a_i} 
                   = \sum_{i=1}^{3}v(a_i)\phi_{a_i} 
                   = 0
        \end{equation}

    \item[(b)] 有限元空间$V_h$可以如下描述,易知为有限维的线性空间。
        \begin{equation}
            V_h = 
            \left\{
                v_h \in C(\overline{\Omega}) :
                v_h \big|_T \in \mathcal{P}_1(T), 
                \forall \ T \in \mathcal{T}_h    
            \right\}
        \end{equation}
        下证 ${\phi_i}$ 构成有限元空间的一组基底,即
        \begin{equation}
            c_1 \phi_1 + c_2 \phi_2 + \cdots + c_{NV} \phi_{NV} = 0
        \end{equation}
        其中 $0$ 为零函数,
        由基函数的局部性 $\phi_i(x_j) = \delta_{ij}$,易得 $c_i=0$,
        因此 ${\phi_i}$ 线性无关。
        \\
        另一方面,对于任意 $v_h \in V_h$,由分片线性插值的性质可知
        \begin{equation}
            v_h = \sum_{i=1}^{NV} v_h(x_i) \phi_i
        \end{equation}
        因此 ${\phi_i}$ 构成 $V_h$ 的一组基底。
\end{enumerate} 

\section*{3. 有限维的赋范线性空间及其线性映射}

由于 $V$ 是有限维的赋范线性空间,由范数的等价性可知

\begin{equation}
    C_1 \Vert v \Vert_{1} \leq \Vert v \Vert_{V} \leq C_2 \Vert v \Vert_{1}
\end{equation}

设 $\{v_i\}_{1\leq i \leq n}$ 为 $V$ 的一组基,则 $\forall v \in V$,
有 $v = \sum_{i=1}^{n}c_i v_i$,并令 $\Vert v\Vert_1 = \sum_{i=1}^{n}|c_i|$

\begin{equation}
    \begin{aligned}
        \frac{\ell(v)}{\Vert v \Vert_V}
        \leq \frac{1}{C_1} \frac{|\ell(v)|}{\Vert v \Vert_{1}}
        &= \frac{1}{C_1} \frac{|\ell(\sum_{i=1}^{n}c_i v_i)|}{\Vert \sum_{i=1}^{n}c_i v_i \Vert_{1}}\\
        &= \frac{1}{C_1} \frac{|\sum_{i=1}^{n}c_i \ell(v_i)|}{\sum_{i=1}^{n}|c_i|}\\
        &\leq \frac{1}{C_1} \frac{\sum_{i=1}^{n}|c_i| |\ell(v_i)|}
            {\sum_{i=1}^{n}|c_i|}\\
        &\leq \frac{1}{C_1} \max_{1\leq i \leq n} |\ell(v_i)| < \infty   
    \end{aligned}
\end{equation}

由此可知 $\Vert \ell \Vert_{V^{*}}$ 有界。

\section*{4. $L^2(\Omega)$ 空间中函数列的相关定理}

由 Cauchy-Schwarz 不等式可得,

\begin{equation}
    \begin{aligned}
        \left(
            \int_{\Omega} (u_n - u) v \text{d}x
        \right)^2
        &\leq \int_{\Omega} (u_n - u)^2 \text{d}x
            \int_{\Omega} v^2 \text{d}x\\
        &= \Vert u_n - u \Vert^2_{L^2(\Omega)} \Vert v \Vert^2_{L^2(\Omega)}
        \rightarrow 0
    \end{aligned}
\end{equation}

该积分收敛到 $0$,是由 $u_n \rightarrow u$ 在 $L^2(\Omega)$ 意义下收敛
和 $\Vert v \Vert_{L^2(\Omega)}< \infty$ 保证的,易得

\begin{equation}
    \lim_{n \rightarrow \infty} \int_{\Omega} u_n v \text{d}x
    = \int_{\Omega} u v \text{d}x
\end{equation}

\section*{5. 绝对连续函数的相关概念}

\begin{enumerate}
    \item[(a)] 对 $\forall \epsilon>0$,有 $\delta>0$,使对 $[0,1]$
                上两两不相交的开区间 $\{(a_i,b_i)\}_{1\leq i \leq n}$ 满足
                $\sum_{i=1}^{n} (b_i - a_i) < \delta$时,有
                \begin{equation}
                    \sum_{i=1}^{n} |f(b_i) - f(a_i)| < \epsilon
                \end{equation}
    \item[(b)] 对 $[0,1]$ 上两两不相交
                且满足 $\sum_{i=1}^{n} (b_i - a_i) < \delta$ 的开区间 $\{(a_i,b_i)\}_{1\leq i \leq n}$ 有
                \begin{equation}
                    \begin{aligned}
                        \sum_{i=1}^{n} |u(b_i) - u(a_i)|
                        &= \sum_{i=1}^{n} \left|
                            \int_{a_i}^{b_i} v(t) \text{d}t
                        \right|\\
                        &\leq \sum_{i=1}^{n} 
                            \int_{a_i}^{b_i} |v(t)| \text{d}t\\
                        &= \int_{\cup_{i=1}^{n}(a_i, b_i) } |v(t)| \text{d}t\\
                    \end{aligned}
                \end{equation} 
                由积分的绝对连续性可证 $u(x)$ 为绝对连续函数。             
\end{enumerate}

\section*{6. 绝对连续函数的相关定理}

令 $f(x) = u(x) v(x)$,由于 $u, v$ 均为 $[0,1]$ 上的绝对连续函数,
因此 $f$ 也是 $[0,1]$ 上的绝对连续函数。

\begin{equation}
    f(1) - f(0) = \int_{0}^{1} f'(x) \text{d}x
\end{equation}

也就是

\begin{equation}
    \begin{aligned}
        u(1)v(1) - u(0)v(0)
        &= \int_{0}^{1} (uv)' \text{d}x\\
        &= \int_{0}^{1} u'v \text{d}x
            + \int_{0}^{1} uv' \text{d}x
    \end{aligned}
\end{equation}

\section*{7. Dirichlet 边界条件的 Poisson 方程}

\begin{enumerate}
    \item[(a)] 令 $a(u,v)=\int_{\Omega}\nabla u \cdot \nabla v \text{d}x$,
                验证其有界性和强制性。
            \begin{equation}
                |a(u,v)|
                \leq \int_{\Omega} |\nabla u \cdot \nabla v| \text{d}x
                \leq \int_{\Omega} |\nabla u| |\nabla v| \text{d}x
                \leq \Vert \nabla u \Vert_{L^2(\Omega)}
                    \Vert \nabla v \Vert_{L^2(\Omega)}
                = |u|_{H^1(\Omega)}
                    |v|_{H^1(\Omega)}
            \end{equation}
            \begin{equation}
                |a(u,u)|
                = \int_{\Omega} |\nabla u|^2 \text{d}x
                = |u|_{H^1(\Omega)}^2
            \end{equation}
            根据 Lax-Milgram 定理可知该问题有唯一解。
    \item[(b)] 由定义知 $A_{ij} = \int_{\Omega} \nabla \phi_i \cdot \nabla \phi_j \text{d}x$ 
    因此 $A_{ij} = A_{ji}$,所以 $A$ 是对称矩阵。\\
    又由 (c) 可知 $\forall \vec{v_h} \neq \vec{0}$,有
    \begin{equation}
        \vec{v_h}^\top A \vec{v_h}
        = \Vert \nabla v_h \Vert^2_{L^2(\Omega)} > 0
    \end{equation}
    因此 $A$ 是正定矩阵。
    \item[(c)] 令 $v_h = \sum_{i=1}^{NV}c_i \phi_i$,代入可得
            \begin{equation}
                \Vert \nabla v_h \Vert^2_{L^2(\Omega)}
                = \int_{\Omega} 
                    \left[\nabla \left(\sum_{i=1}^{NV}c_i \phi_i\right)\right]^2 \text{d}x
                = \int_{\Omega} 
                    \left( \sum_{i=1}^{NV}c_i \nabla \phi_i \right)^2 \text{d}x
            \end{equation}
            \begin{equation}
                \vec{v_h}^\top A\vec{v_h}
                =
                \begin{pmatrix}
                    c_1 & c_2 & \cdots & c_{NV}
                \end{pmatrix}
                \begin{pmatrix}
                    \int_{\Omega} \nabla \phi_1 \cdot \nabla \phi_1 \text{d}x &
                    \int_{\Omega} \nabla \phi_1 \cdot \nabla \phi_2 \text{d}x &
                    \cdots &
                    \int_{\Omega} \nabla \phi_1 \cdot \nabla \phi_{NV} \text{d}x\\
                    \int_{\Omega} \nabla \phi_2 \cdot \nabla \phi_1 \text{d}x &
                    \int_{\Omega} \nabla \phi_2 \cdot \nabla \phi_2 \text{d}x &
                    \cdots &
                    \int_{\Omega} \nabla \phi_2 \cdot \nabla \phi_{NV} \text{d}x\\
                    \vdots & \vdots & \ddots & \vdots\\
                    \int_{\Omega} \nabla \phi_{NV} \cdot \nabla \phi_1 \text{d}x &
                    \int_{\Omega} \nabla \phi_{NV} \cdot \nabla \phi_2 \text{d}x &
                    \cdots &
                    \int_{\Omega} \nabla \phi_{NV} \cdot \nabla \phi_{NV} \text{d}x
                \end{pmatrix}
                \begin{pmatrix}
                    c_1\\ c_2\\ \vdots\\ c_{NV}
                \end{pmatrix}
            \end{equation}  
            比较可知,上面二者相等。
    
\end{enumerate}

\section*{8. Robin 边界条件下的Poisson 方程}

\begin{enumerate}
    \item[(a)] 该问题的有限元方法可以由以下推导得到
    \begin{equation}
        \begin{gathered}
            -\int_{\Omega} (\Delta u) v \text{d}S = \int_{\Omega} f v \text{d}S\\
            -\int_{\partial\Omega} (\nabla u \cdot \vec{n}) v \text{d}l
            + \int_{\Omega} \nabla u \cdot \nabla v \text{d}S
            =\int_{\Omega} f v \text{d}S\\
            -\int_{\partial\Omega} \frac{\partial u}{\partial n} v \text{d}l
            + \int_{\Omega} \nabla u \cdot \nabla v \text{d}S
            =\int_{\Omega} f v \text{d}S\\
            \int_{\partial\Omega} u v \text{d}l
            + \int_{\Omega} \nabla u \cdot \nabla v \text{d}S
            =\int_{\Omega} f v \text{d}S
        \end{gathered}
    \end{equation}
    用分片线性函数空间 $V_h$ 近似,则有限元问题为:
    \begin{equation}
        \int_{\Omega} \nabla u_h \cdot \nabla v_h \text{d}S
            =\int_{\Omega} f v_h \text{d}S
            - \int_{\partial\Omega} u_h v_h \text{d}l 
    \end{equation}
    \item[(b)] 定义 Robin 边界条件下的能量泛函为
    \begin{equation}
        E(v) = \frac{1}{2} \int_{\Omega} |\nabla v|^2 \text{d}S
            + \frac{1}{2} \int_{\partial\Omega} v^2 \text{d}l
            - \int_{\Omega} f v \text{d}S
    \end{equation}
    该问题的 Ritz 方法为
    \begin{equation}
        u_h = \arg \min_{v_h \in V_h} E(v_h)
    \end{equation}
    \item[(c)] 定义 $\varphi(\lambda) = E(u + \lambda v)$,接下来我们
                把 $\varphi(\lambda)$ 对 $\lambda$ 求导。
            \begin{equation}
                \begin{aligned}
                    \varphi'(\lambda)
                    &= \frac{1}{2} \int_{\Omega} 2\nabla (u + \lambda v) \cdot \nabla v \text{d}S
                        + \frac{1}{2} \int_{\partial\Omega} 2(u + \lambda v)v \text{d}l
                        - \int_{\Omega} f v \text{d}S\\
                    &= \int_{\Omega} \nabla (u + \lambda v) \cdot \nabla v \text{d}S
                        + \int_{\partial\Omega} (u + \lambda v)v \text{d}l
                        - \int_{\Omega} f v \text{d}S
                \end{aligned}
            \end{equation}
            由 $u$ 是该问题的解可知 $\varphi'(0) = 0$,也就是
            \begin{equation}
                \int_{\Omega} \nabla u \cdot \nabla v \text{d}S
                    + \int_{\partial\Omega} uv \text{d}l
                    = \int_{\Omega} f v \text{d}S
            \end{equation}
            所以上述的有限元方法和 Ritz 方法是等价的。
\end{enumerate}

% ===============================================

\printbibliography

\end{document}