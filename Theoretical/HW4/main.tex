\documentclass[a4paper]{article}
\usepackage{ctex}
\usepackage[affil-it]{authblk}
\usepackage[backend=bibtex,style=numeric]{biblatex}
\usepackage{amsmath,amsthm,amssymb,amsfonts}
\usepackage{bm, mathrsfs}
\usepackage{pifont}
\usepackage{graphicx, subcaption}

\usepackage{geometry}
\geometry{margin=1.5cm, vmargin={0pt,1cm}}
\setlength{\topmargin}{-1cm}
\setlength{\paperheight}{29.7cm}
\setlength{\textheight}{25.3cm}

\addbibresource{citation.bib}

\begin{document}
% =================================================
\title{有限元方法 2025 秋冬作业四}

\author{曾申昊 3220100701
  \thanks{邮箱: \texttt{3097714673@qq.com}
                            \texttt{或} \texttt{3220100701@zju.edu.cn}}}
\affil{电子科学与技术 2202,浙江大学 }


\date{更新时间: \today}

\maketitle


% ============================================
\section*{1. Sobolev 空间和弱导数的相关性质}

\begin{enumerate}
    \item[(a)] 取 $\phi \in C_c^{\infty}(-1,1)$,根据定义有
            \begin{equation}
                \begin{aligned}
                    \int_{-1}^{1} u(x) \phi'(x) \text{d}x
                    &= \int_{-1}^{0} (1+x) \phi'(x) \text{d}x + \int_{0}^{1} (1-x) \phi'(x) \text{d}x\\
                    &= \Big. (1+x) \phi(x) \Big|_{-1}^{0} - \int_{-1}^{0}1\cdot \phi(x) \text{d}x + \Big. (1-x) \phi(x) \Big|_{0}^{1} - \int_{0}^{1}(-1)\cdot \phi(x) \text{d}x\\
                    &= -\int_{-1}^{1}u'(x)\phi(x) \text{d}x    
                \end{aligned}
            \end{equation}
    其中定义弱导数 $u'(x) = \begin{cases}1, & x \in (-1,0)\\ -1, & x \in (0,1)\end{cases}$。
    \item[(b)] 取 $\phi \in C_c^{\infty}(-1,1)$,根据定义有
            \begin{equation}
                \begin{aligned}
                    \int_{-1}^{1} u''(x) \phi(x) \text{d}x
                    &= -\int_{-1}^{1} u'(x) \phi'(x) \text{d}x\\
                    &= -\int_{-1}^{0} 1 \cdot \phi'(x) \text{d}x - \int_{0}^{1} (-1) \cdot \phi'(x) \text{d}x\\
                    &= -\Big. \phi(x) \Big|_{-1}^{0} + \Big. \phi(x) \Big|_{0}^{1}\\
                    &= -2\phi(0)
                \end{aligned}
            \end{equation}
            利用反证法,现在假设 $u'' \in L^1(-1,1)$,
            取 $\phi_n \in C_c^{\infty}(-1,1)$ 满足 
            $\phi_n(0)=1$,支集在 $[-\frac{1}{n},\frac{1}{n}]$上,
            且 $|\varphi_n|\leq 1$,则有
            \begin{equation}
                \int_{-1}^{1} u''(x) \phi_n(x) \text{d}x
                = - 2\phi_n(0)
                = -2
            \end{equation}
            另一方面,根据积分的绝对连续性,有  
            \begin{equation}
                \begin{aligned}
                    \left|\int_{-1}^{1} u''(x) \phi_n(x) \text{d}x\right|
                    &\leq \int_{-1}^{1} |u''(x)||\phi_n(x)| \text{d}x\\
                    &\leq \int_{[-\frac{1}{n},\frac{1}{n}]} |u''(x)| \text{d}x
                    \to 0
                \end{aligned}
            \end{equation}
            当 $n \to \infty$ 时产生矛盾,因此 $u'' \notin L^2(-1,1)$。
    \item[(c)] 证明下面二者定义等价:
            \begin{equation}
                \begin{cases}
                    v(x) = v(0) + \displaystyle\int_{0}^{x} v'(t) \text{d}t\\
                    \displaystyle\int_{0}^{1} (v'(t))^2 \text{d}t < \infty\\
                    v'(x) = \displaystyle\lim_{h \to 0}\frac{v(x+h) - v(x)}{h}
                \end{cases}
                \Longleftrightarrow 
                \quad
                \begin{cases}
                    \displaystyle\int_{0}^{1} (v(t))^2 \text{d}t < \infty\\
                    \displaystyle\int_{0}^{1} (v'(t))^2 \text{d}t < \infty\\
                    \displaystyle\int_{0}^{1} v'(t) \phi(t) \text{d}t 
                    = -\displaystyle\int_{0}^{1} v(t) \phi'(t) \text{d}t
                \end{cases}
            \end{equation}
            \begin{itemize}
                \item “$\Rightarrow$”\\
                    根据 Minkowski 不等式和 Cauchy-Schwarz 不等式,有
                    \begin{equation}
                        \begin{aligned}
                            \Vert v(x) \Vert_{L^2(0,1)}
                            &\leq \Vert v(x) - v(0) \Vert_{L^2(0,1)} + \Vert v(0) \Vert_{L^2(0,1)}\\
                            &= \left(\int_{0}^{1}\left(\int_{0}^{t} v'(s) \text{d}s\right)^2\text{d}t\right)^{\frac{1}{2}}
                                + |v(0)|\\
                            &\leq \left(\int_{0}^{1}
                                \left[
                                    \int_{0}^{t} (v'(s))^2 \text{d}s
                                    \cdot \int_{0}^{t} 1^2 \text{d}s
                                \right]
                                \text{d}t\right)^{\frac{1}{2}}
                                + |v(0)|\\
                            &\leq \left(\int_{0}^{1}
                                C_0t \  
                                \text{d}t\right)^{\frac{1}{2}}
                                + |v(0)|\\
                            &\leq \sqrt{\frac{C_0}{2}}
                                + |v(0)| < \infty
                        \end{aligned}
                    \end{equation}
                    另一方面,经典导数 $v'(x)$ 自然满足弱导数定义。
                \item “$\Leftarrow$”\\
                    令 $u(x) = \int_{0}^{x}v'(t)\text{d}t$,
                    则 $u(x)\in \text{AC}[0,1]$ 且 $u'(x) \overset{a.e.}{=} v'(x)$。
                    令 $w(x)=u(x)-v(x)$,计算 $w(x)$ 的弱导数有
                    \begin{equation}
                        \begin{aligned}
                            \int_{0}^{1} w(x) \phi'(x) \text{d}x
                            &= \int_{0}^{1} (u(x)-v(x)) \phi'(x) \text{d}x\\
                            &= \int_{0}^{1} u(x) \phi'(x) \text{d}x - \int_{0}^{1} v(x) \phi'(x) \text{d}x\\    
                            &= -\int_{0}^{1} u'(x) \phi(x) \text{d}x + \int_{0}^{1} v'(x) \phi(x) \text{d}x\\
                            &= 0
                        \end{aligned}
                    \end{equation}
                    由于 $\phi(x)$ 的任意性可得, $w'(x)\overset{a.e.}{=} 0$,即
                    $w(x) \overset{a.e.}{=} C$。因此有 $w(x)\in \text{AC}[0,1]$,
                    故 $v(x) = u(x) - w(x)$ 也是绝对连续函数,其经典导数几乎处处存在。
            \end{itemize}
\end{enumerate}

\section*{2. 二维区域 $\Omega$ 上的向量场}

\begin{enumerate}
    \item[(a)] 
        \begin{itemize}
            \item “$\Rightarrow$”\\
                由 $\varepsilon(\vec{u}) = O$ 可得
                \begin{equation}
                    \left\{
                    \begin{aligned}
                        \partial_1 u_1 + \partial_1 u_1 &= 0,\\
                        \partial_2 u_2 + \partial_2 u_2 &= 0,\\
                        \partial_1 u_2 + \partial_2 u_1 &= 0.
                    \end{aligned}
                    \right.
                \end{equation}
                由前两个方程可得 $u_1 = f(y)$ 且 $u_2 = g(x)$,将其代入第三个方程可得
                \begin{equation}
                    \frac{\text{d}g(x)}{\text{d}x} + \frac{\text{d}f(y)}{\text{d}y} = 0.
                \end{equation}
                根据 $x$ 和 $y$ 的独立性可令
                \begin{equation}
                    \frac{\text{d}g(x)}{\text{d}x} = a \qquad \frac{\text{d}f(y)}{\text{d}y} = -a
                \end{equation}
                其中 $a$ 为常数。由此可得
                \begin{equation}
                    \left\{
                    \begin{aligned}
                        u_1(x,y) &= -ay + b,\\  
                        u_2(x,y) &= ax + c,
                    \end{aligned}
                    \right.
                \end{equation}
                其中 $b,c$ 为常数。
            \item “$\Leftarrow$”\\
                将 $\vec{u}(x,y)=(-ay+b,ax+c)$
                代入 $\varepsilon(\vec{u})_{ij}$ 的定义可得
                \begin{equation}
                    \left\{
                    \begin{aligned}
                        \varepsilon(\vec{u})_{11} &= \frac{1}{2}(\partial_1 u_1 + \partial_1 u_1) = 0,\\
                        \varepsilon(\vec{u})_{22} &= \frac{1}{2}(\partial_2 u_2 + \partial_2 u_2) = 0,\\
                        \varepsilon(\vec{u})_{12} &= \varepsilon(\vec{u})_{21} 
                        = \frac{1}{2}(\partial_1 u_2 + \partial_2 u_1)
                        = \frac{1}{2}(a - a) = 0.
                    \end{aligned}
                    \right.
                \end{equation}
                由此可知 $\varepsilon(\vec{u}) = O$.
        \end{itemize}
    \item[(b)]
        \begin{itemize}
            \item 预紧集合定义\\
                $(E,d)$ 是一个 Banach 空间,$S \subset  E$ 被称为是一个预紧集合,
                如果满足对 $\forall \varepsilon > 0$,
                $S$ 能被有限个半径为 $\varepsilon$ 的开球覆盖。
            \item 紧算子定义\\
                一个算子 $T: E \to F$,
                其中 $E,F$ 均为 Banach空间,被称为是紧算子的,
                如果作用于 $E$ 中的单位球得到的像集是预紧的。
        \end{itemize}
    \item[(c)] 类似讲义中 Poincaré 不等式的证明过程。
                假设Korn 不等式不成立,则存在 $\{ \vec{u}_n \} \subset H_0^1(\Omega)$ 满足
                \begin{equation}
                    \Vert \vec{u}_n \Vert_{L^2(\Omega)} \geq
                    n \Vert \varepsilon(\vec{u}_n) \Vert_{L^2(\Omega)}.
                \end{equation}
                单位化 $\vec{u}_n$,
                令 $\vec{v}_n = \frac{\vec{u}_n}{\Vert \vec{u}_n \Vert_{L^2(\Omega)}}$,则有
                \begin{equation}
                    \Vert \varepsilon(\vec{v}_n) \Vert_{L^2(\Omega)}
                    \leq \frac{1}{n} 
                \end{equation}
                由此可得 $\vec{v}_n$ 是 $H_0^1(\Omega)$ 中的有界序列,
                也是 $L^2(\Omega)$ 中的预紧集合。
                因此存在子列 $\{\vec{v}_{n_k}\}$
                使得 
                \begin{gather}
                    \Vert \vec{v}_{n_k} - \vec{v}\Vert_{L^2(\Omega)} \to 0\\
                    \Vert \vec{v}_{n_k} - \vec{w}\Vert_{H^1(\Omega)} \to 0
                \end{gather}
                并且可知 $v \overset{a.e.}{=} w$。取 $n\to \infty$ 后可以得到
                $\varepsilon(\vec{v}) = O$,由 (a) 小问可知
                \begin{equation}
                    \left\{
                    \begin{aligned}
                        v_1(x,y) &= -ay + b,\\  
                        v_2(x,y) &= ax + c,
                    \end{aligned}
                    \right.
                \end{equation}
                由于 $\vec{v} \in H_0^1(\Omega)$,所以 $a = b = c = 0$,即 $\vec{v} = O$。
                这与 $\Vert \vec{v} \Vert_{L^2(\Omega)} = 1$ 矛盾。
\end{enumerate}

\section*{3. 三角形 $T$ 上 Lagrange 插值的误差估计}

\begin{enumerate}
    \item[(a)]  写出基于 $T$ 的顶点的一次 Lagrange 插值
                \begin{equation}
                    I_T u =  \sum_{i=1}^{3} u(\bm{a_i}) \phi_i
                \end{equation}
                利用多点 Taylor 展开式,有
                \begin{equation}
                    u(\bm{a_i}) = u(\bm{x}) + \nabla u(\bm{x}) \cdot (\bm{a_i} - \bm{x})
                    + \int_{0}^{1} (\bm{a_i} - \bm{x})^T \nabla^2 u(\bm{x} + t(\bm{a_i} - \bm{x}))
                    (\bm{a_i} - \bm{x})(1-t) \text{d}t
                \end{equation}
                将其代入 $I_T u$ 可得
                \begin{equation}
                    \begin{aligned}
                        I_T u 
                        &=  \sum_{i=1}^{3} u(\bm{a_i}) \phi_i\\
                        &=  \sum_{i=1}^{3} \left( u(\bm{x}) + \nabla u(\bm{x}) 
                        \cdot (\bm{a_i} - \bm{x}) + \int_{0}^{1} (\bm{a_i} - \bm{x})^T \nabla^2 u(\bm{x} + t(\bm{a_i} - \bm{x})) (\bm{a_i} - \bm{x})(1-t) \text{d}t \right) \phi_i\\
                        &= u(\bm{x}) \sum_{i=1}^{3} \phi_i
                        + \nabla u(\bm{x}) \cdot \sum_{i=1}^{3} (\bm{a_i} - \bm{x}) \phi_i
                        + \sum_{i=1}^{3}\phi_i \left( \int_{0}^{1} (\bm{a_i} - \bm{x})^T \nabla^2 u(\bm{x} + t(\bm{a_i} - \bm{x})) (\bm{a_i
                        } - \bm{x})(1-t) \text{d}t \right) \\
                        &=u(\bm{x})+\sum_{i=1}^{3}\phi_i \left( \int_{0}^{1} (\bm{a_i} - \bm{x})^T \nabla^2 u(\bm{x} + t(\bm{a_i} - \bm{x})) (\bm{a_i
                        } - \bm{x})(1-t) \text{d}t \right)
                    \end{aligned}
                \end{equation}
                \begin{itemize}
                    \item 证明 $\Vert u-I_T u  \Vert_{L^p(T)} \leq C_1 h_T^2 |u|_{W^{2,p}(T)}$
                        \begin{equation}
                            \begin{aligned}
                                \Vert u-I_T u  \Vert_{L^p(T)}
                                &=\left\| \sum_{i=1}^{3}\phi_i \left( \int_{0}^{1
                                } (\bm{a_i} - \bm{x})^T \nabla^2 u(\bm{x} + t(\bm{a_i} - \bm{x})) (\bm{a_i
                                } - \bm{x})(1-t) \text{d}t \right) \right\| _{L^p(T)}\\
                                &\leq \left\| \sum_{i=1}^{3}\left( \int_{0}^{1
                                } (\bm{a_i} - \bm{x})^T \nabla^2 u(\bm{x} + t(\bm{a_i} - \bm{x})) (\bm{a_i
                                } - \bm{x})(1-t) \text{d}t \right) \right\| _{L^p(T)}\\
                                &\leq h_T^2\sum_{i=1}^{3}\left( \int_{0}^{1
                                } \left\|
                                    \nabla^2 u(\bm{x} + t(\bm{a_i} - \bm{x})) 
                                \right\|_{L^p(T)}
                                (1-t) \text{d}t \right)
                                \label{3-a-1}
                            \end{aligned}
                        \end{equation}
                        对 $L^p(T)$ 进行估计
                        \begin{equation}
                            \begin{aligned}
                                \left\| \nabla^2 u(\bm{x} + t(\bm{a_i} - \bm{x})) \right\|_{L^p(T)}
                                &= \left(\int_{T} \left( \nabla^2 u(\bm{x} + t(\bm{a_i} - \bm{x})) \right)^p \text{d}\bm{x}\right)^{\frac{1}{p}}\\
                                &= \left(\int_{T} \left( \nabla^2 u(\bm{z}) \right)^p (1-t)^{-2}\text{d}\bm{z}\right)^{\frac{1}{p}}\\
                                &= (1-t)^{-\frac{2}{p}}\left(\int_{T} \left( \nabla^2 u(\bm{z}) \right)^p \text{d}\bm{z}\right)^{\frac{1}{p}}
                                \label{L^p(T)估计}
                            \end{aligned}
                        \end{equation}      
                        将式 (\ref{L^p(T)估计}) 代入式 (\ref{3-a-1}) 可得
                        \begin{equation}
                            \begin{aligned}
                                \Vert u-I_T u  \Vert_{L^p(T)}
                                &\leq 3h_T^2\left( \int_{0}^{1
                                } (1-t)^{1-\frac{2}{p}} \text{d}t \right) |u|_{W^{2,p}(T)}\\
                                &= C_1 h_T^2 |u|_{W^{2,p}(T)}
                            \end{aligned}
                        \end{equation}              
                    \item 证明 $|u-I_T u|_{W^{1,p}(T)} \leq C_2 h_T |u|_{W^{2,p}(T)}$                    
                        \begin{equation}
                            \begin{aligned}
                                \nabla I_T u - \nabla u
                                &= \nabla \left( \sum_{i=1}^{3}\phi_i \left( \int_{0}^{1
                                } (\bm{a_i} - \bm{x})^T \nabla^2 u(\bm{x} + t(\bm{a_i} - \bm{x})) (\bm{a_i
                                } - \bm{x})(1-t) \text{d}t \right) \right)\\
                                &= \sum_{i=1}^{3} \nabla \phi_i \left( \int_{0}^{1
                                } (\bm{a_i} - \bm{x})^T \nabla^2 u(\bm{x} + t(\bm{a_i} - \bm{x})) (\bm{a_i
                                } - \bm{x})(1-t) \text{d}t \right)\\
                                &\quad + \sum_{i=1}^{3}\phi_i 
                                \nabla \left( \int_{0}^{1
                                } (\bm{a_i} - \bm{x})^T \nabla^2 u(\bm{x} + t(\bm{a_i} - \bm{x})) (\bm{a_i
                                } - \bm{x})(1-t) \text{d}t \right)\\
                                &= \sum_{i=1}^{3} \nabla \phi_i \left( \int_{0}^{1
                                } (\bm{a_i} - \bm{x})^T \nabla^2 u(\bm{x} + t(\bm{a_i} - \bm{x})) (\bm{a_i
                                } - \bm{x})(1-t) \text{d}t \right)
                            \end{aligned}
                        \end{equation}
                        \begin{equation}
                            \begin{aligned}
                                |u-I_T u|_{W^{1,p}(T)}
                                &= \left\|  \left( \sum_{i=1}^{3}\nabla\phi_i \left( \int_{0}^{1
                                } (\bm{a_i} - \bm{x})^T \nabla^2 u(\bm{x} + t(\bm{a_i} - \bm{x})) (\bm{a_i
                                } - \bm{x})(1-t) \text{d}t \right) \right) \right\| _{L^p(T)}\\
                                &= \sum_{i=1}^{3}\frac{1}{d_i} \left( \int_{0}^{1
                                } \lVert \nabla^2 u(\bm{x} + t(\bm{a_i} - \bm{x})) \rVert _{L^p(T)}
                                (1-t) \text{d}t \right)\\
                                &\leq \frac{3h_T}{\min\{d_1,d_2,d_3\}}h_T \left( \int_{0}^{1
                                } (1-t)^{1-\frac{2}{p}} \text{d}t \right) |u|_{W^{2,p}(T)}\\
                                &= C_2 h_T |u|_{W^{2,p}(T)}
                            \end{aligned}
                        \end{equation}
                \end{itemize}
    \item[(b)]  $C_1$ 的一个明确上界为
                \begin{equation}
                    \begin{aligned}
                        C_1 &= 3\int_{0}^{1
                        } (1-t)^{1-\frac{2}{p}} \text{d}t\\
                        &=\frac{3}{2-\frac{2}{p}}
                    \end{aligned}
                \end{equation}
                当 $\frac{h_T}{\min\{d_1,d_2,d_3\}}$ 较大时,
                也就是三角形 $T$ 较为扁平时,$C_2$ 较大;
                反之当 $\frac{h_T}{\min\{d_1,d_2,d_3\}}$ 较小时,
                也就是三角形 $T$ 较为匀称,即接近等边三角形时,$C_2$ 较小。
    \item[(c)]  
                \begin{itemize}
                    \item 证明 $\Vert u-I_T u\Vert_{L^2(T)} \leq C_3 h^3_T |u|_{H^3(T)}$\\
                        根据 Bramble-Hilbert 定理有
                        \begin{equation}
                            \Vert \hat{u}-\hat{I}_T\hat{u}\Vert_{L^2(\hat{T})} 
                            \leq \tilde{C}_1 |\hat{u}|_{H^3(\hat{T})}  
                        \end{equation}
                        其中 $\hat{T}$ 代表仿射变换后的参考三角形。
                        根据仿射变换的性质,有
                        \begin{equation}
                            \begin{aligned}
                                \Vert u-I_Tu\Vert_{L^2(T)} 
                                &=|\det(B)|^{\frac{1}{2}} \Vert \hat{u}-\hat{I}\hat{u}\Vert_{L^2(\hat{T})} \\
                                &\leq \tilde{C}_1 |\det(B)|^{\frac{1}{2}} |\hat{u}|_{H^3(\hat{T})}\\
                                &\leq \tilde{C}_1 |\det(B)|^{\frac{1}{2}} \tilde{C}_2 \Vert B\Vert^3 |\det(B)|^{-\frac{1}{2}}|u|_{H^3(T)}\\
                                &\leq \tilde{C}_1 \tilde{C}_2 \tilde{C}_3^3 h_T^3 |u|_{H^3(T)}
                            \end{aligned}
                        \end{equation}
                        最后一个不等式是因为三角形 $T$ 是形状规则的,
                        故存在常数 $\tilde{C}_3$ 使得 $\Vert B\Vert \leq \tilde{C}_3 h_T$。
                        令 $C_3 = \tilde{C}_1 \tilde{C}_2 \tilde{C}_3^3$ 即证毕。
                        \item 证明 $|u-I_T u|_{H^1(T)} \leq C_4 h^2_T |u|_{H^3(T)}$\\
                            根据讲义中的 \textbf{Lemma 1.2} 和 \textbf{Lemma 1.3} 有
                            \begin{equation}
                                \begin{aligned}
                                        |v-I_{T}v|_{H^{1}(T)}
                                        & \leq|\det (B)|^{\frac{1}{2}}\|B^{-1}\||\widehat{v-I_{T}v}|_{H^{1}(\widehat{T})} \\
                                        & =|\det (B)|^{\frac{1}{2}}\|B^{-1}\||\hat{v}-I_{\widehat{T}}\hat{v}|_{H^1(\widehat{T})} \\
                                        & =|\det (B)|^{\frac{1}{2}}\|B^{-1}\||\hat{v}-p-I_{\widehat{T}}(\hat{v}-p)|_{H^1(\widehat{T})} \\
                                        & \leq \tilde{C}_4|\det (B)|^{\frac{1}{2}}\|B^{-1}\|\|\hat{v}-p\|_{H^{3}(\widehat{T})}
                                \end{aligned}
                            \end{equation}
                            根据 $p\in\mathcal{P}_m$ 的任意性,有
                            \begin{equation}
                                \begin{aligned}
                                |v-I_{T}v|_{H^{1}(T)} 
                                & \leq \tilde{C}_4|\det (B)|^{\frac{1}{2}}\|B^{-1}\||\hat{v}|_{H^{3}(\widehat{T})} \\
                                & \leq \tilde{C}_4\|B^{-1}\|\|B\|^{3}|v|_{H^{3}(T)} \\
                                & \leq \tilde{C}_4\left(\frac{h_{\widehat{T}}}{\rho_T}\right)h_{T}^{2}|v|_{H^{3}(T)}.
                                \end{aligned}
                            \end{equation}
                            令 $C_4 = \tilde{C}_4 \frac{h_{\hat{T}}}{\rho_T}$ 即证毕。
                \end{itemize}
\end{enumerate}

\section*{4. $\mathbb{R}^2$ 上的空间 $H^k(\mathbb{R}^2)$ 的相关不等式}

\begin{enumerate}
    \item[(a)] 根据讲义的定义有 $\hat{u}(\omega)
                = (\mathscr{F}u(\omega))=\int_{\mathbb{R}^n}u(x)e^{-2\pi i \omega \cdot x }\text{d}x$
                由此可得
                \begin{equation}
                    \left\{
                    \begin{aligned}
                        |u(x)|_{H^1(\mathbb{R}^2)}
                        &= \Vert \nabla u(x) \Vert_{L^2(\mathbb{R}^2)}
                        = 2 \pi \Vert \omega\hat{u}(\omega) \Vert_{L^2(\mathbb{R}^2)}\\
                        |u(x)|_{H^2(\mathbb{R}^2)}
                        &= \Vert \nabla^2 u(x) \Vert_{L^2(\mathbb{R}^2)}
                        = 4 \pi^2 \Vert \omega^2\hat{u}(\omega) \Vert_{L^2(\mathbb{R}^2)}\\
                    \end{aligned}
                    \right.
                \end{equation}
                \begin{equation}
                    \left\{
                    \begin{aligned}
                        \Vert u(x) \Vert_{L^2(\mathbb{R}^2)} 
                        &= \Vert \hat{u}(\omega) \Vert_{L^2(\mathbb{R}^2)}\\
                        \Vert u(x) \Vert_{H^1(\mathbb{R}^2)}
                        &= \left(
                            \Vert \hat{u}(\omega) \Vert^2_{L^2(\mathbb{R}^2)}
                            + \Vert 2 \pi \omega\hat{u}(\omega) \Vert^2_{L^2(\mathbb{R}^2)}
                        \right)^{\frac{1}{2}}\\
                        \Vert u(x) \Vert_{H^2(\mathbb{R}^2)}
                        &= \left(
                            \Vert \hat{u}(\omega) \Vert^2_{L^2(\mathbb{R}^2)}
                            + 2\Vert 2 \pi \omega\hat{u}(\omega) \Vert^2_{L^2(\mathbb{R}^2)}
                            + \Vert 4 \pi^2 \omega^2\hat{u}(\omega) \Vert^2_{L^2(\mathbb{R}^2)}
                        \right)^{\frac{1}{2}}
                    \end{aligned}
                    \right.
                \end{equation}
    \item[(b)] 根据 $\Vert u \Vert_{H^2(\mathbb{R}^2)}$ 的定义
                \begin{equation}
                    \begin{aligned}
                        \Vert u(x) \Vert_{H^2(\mathbb{R}^2)}
                        &= \left(
                            \Vert \hat{u}(\omega) \Vert^2_{L^2(\mathbb{R}^2)}
                            + 2\Vert 2 \pi \omega\hat{u}(\omega) \Vert^2_{L^2(\mathbb{R}^2)}
                            + \Vert 4 \pi^2 \omega^2\hat{u}(\omega) \Vert^2_{L^2(\mathbb{R}^2)}
                        \right)^{\frac{1}{2}}\\
                        &= \left(
                            \int_{\mathbb{R}^2} 
                            (1+4\pi^2\omega^2)^2
                            \hat{u}^2(\omega) \text{d}\omega
                        \right)^{\frac{1}{2}}\\
                        &=\Vert (1+4\pi^2\omega^2)
                            \hat{u}(\omega) \Vert_{L^2(\mathbb{R}^2)}\\
                        &\leq
                        \Vert \hat{u}(\omega) \Vert_{L^2(\mathbb{R}^2)}
                        + \Vert 4 \pi^2 \omega^2\hat{u}(\omega) \Vert_{L^2(\mathbb{R}^2)}\\
                        &= \Vert u(x) \Vert_{L^2(\mathbb{R}^2)}
                            + |u(x)|_{H^2(\mathbb{R}^2)}
                    \end{aligned}
                \end{equation}
    \item[(c)]  根据 Fourier 逆变换有
                \begin{equation}
                    \begin{aligned}
                        |u(x)| &= \left|\int_{\mathbb{R}^2}
                            \hat{u}(\omega)e^{2\pi i \omega \cdot x} \text{d}\omega
                        \right|\\
                        &\leq \int_{\mathbb{R}^2} |\hat{u}(\omega)|\text{d}\omega\\
                        &\leq \left(
                            \int_{\mathbb{R}^2} (1+4\pi^2\omega^2)^{-2} \text{d}\omega
                        \right)^{\frac{1}{2}}
                        \cdot
                        \left(\int_{\mathbb{R}^2} 
                            (1+4\pi^2\omega^2)^2
                            \hat{u}^2(\omega) \text{d}\omega
                        \right)^{\frac{1}{2}}\\
                        &= C \Vert u(x) \Vert_{H^2(\mathbb{R}^2)}
                    \end{aligned}
                \end{equation}
                两边取最大值可得
                \begin{equation}
                    \Vert u(x) \Vert_{L^{\infty}(\mathbb{R}^2)}
                    \leq C \Vert u(x) \Vert_{H^ 2(\mathbb{R}^2)}
                \end{equation}
\end{enumerate}

% ===============================================

\printbibliography

\end{document}